\documentclass[11]{report}
\lstset{breaklines=true}
\setcounter{tocdepth}{3}
\setcounter{secnumdepth}{3}

%\pagestyle{fancy}
\author{Sandeep Gupta\\
  Network Dynamics and Simulation Science Laboratory\\
  Biocomplexity Institute of Virginia Tech\\
  Virginia Tech\\
  Blacksburg, VA, 24061\\
  \texttt{sandeep@vbi.vt.edu}}




\definecolor{codegreen}{rgb}{0,0.6,0}
\definecolor{codegray}{rgb}{0.2,0.2,0.2}
\definecolor{codepurple}{rgb}{0.58,0,0.82}
\definecolor{backcolour}{rgb}{0.99,0.98,0.98}
\definecolor{backframecolour}{rgb}{0.99,0.96,0.97}
\definecolor{backcodepy}{rgb}{0.75,0.70,0.70}


%%%%%%%%%%%%%%%%%%%%%%%%%%%%%%%%%%%%%%%%%%%%%%%%%%%%%%%%%%%%%%%%%%%%%%%%%%%%%%%%%%%%%%%%%%%%%%%%%%%%
%% Latex enviorment
%%%%%%%%%%%%%%%%%%%%%%%%%%%%%%%%%%%%%%%%%%%%%%%%%%%%%%%%%%%%%%%%%%%%%%%%%%%%%%%%%%%%%%%%%%%%%%%%%%%%
%% \definecolor{keywords}{HTML}{8A4A0B}
%% \definecolor{background}{HTML}{EEEEEE}
%% \lstset{language=[LaTeX]Tex,
%%     keywordstyle=\color{keywords},
%%     keywordstyle=[2]\color{blue!40},
%%     keywordstyle=[3]\color{brown!40},
%%     keywordstyle=[4]\color{brown!40},
%%     keywordstyle=[5]\color{red!40},
%%     keywordstyle=[6]\color{black!40},
%%     basicstyle=\ttfamily\scriptsize,
%%     showstringspaces=false,
%%     frameround=ftff,
%%     frame=lines,
%%     morekeywords={RequirePackage,ProvidesPackage},
%% %    keywords=[2]{person,household,activities,location,activity,work,location,school,location,home,location,blkgrp,tract,county,state,persons,collection,households,contact,contacts,
%% %experiment,infection,infections,replicate,cell,disease, model,health,status,health,profile,social,distancing,av,treatment,vaccination,close,work,close,school,intervention,policy,intervention,policies, infected,blockgroup,replicates
%% %},
%% %    keywords=[3]{type,of,member,lives,at,performs,between,duration,overlaps,studies,applies,computes,models,with,seed,affects,models,infectivity,susseptibility,transmisibilaty,infector,infectee,day,infection},
%%  %   keywords=[4]{has,age,gender,zipcode,scheduled,blkgrp},
%%   %  keywords=[5]{scalar,time,range},
%%    % keywords=[6]{create,from,where,get,overlap,graph},
%%     backgroundcolor=\color{background},
%%     numbers=right, numberstyle=\tiny, breaklines=true, numbersep=5pt,escapechar=\%,
%% }
%%%%%%%%%%%%%%%%%%%%%%%%%%%%%%%%%%%%%%%%%%%%%%%%%%%%%%%%%%%%%%%%%%%%%%%%%%%%%%%%%%%%%%%%%%%%%%%%%%%%


%%%%%%%%%%%%%%%%%%%%%%%%%%%%%%%%%%%%%%%%%%%%%%%%%%%%%%%%%%%%%%%%%%%%%%%%%%%%%%%%%%%%%%%%%%%%%%%%%%%%
%% Python enviorment
%%%%%%%%%%%%%%%%%%%%%%%%%%%%%%%%%%%%%%%%%%%%%%%%%%%%%%%%%%%%%%%%%%%%%%%%%%%%%%%%%%%%%%%%%%%%%%%%%%%%
%\DeclareFixedFont{\ttb}{T1}{txtt}{bx}{n}{10} % for bold
%\DeclareFixedFont{\ttm}{T1}{txtt}{m}{n}{10}  % for normal

\definecolor{deepblue}{rgb}{0,0.0,0.5}
\definecolor{deepred}{rgb}{0.6,0,0}
\definecolor{deepgreen}{rgb}{0,0.5,0}
\newcommand\pythonstyle{\lstset{
language=Python,
basicstyle=\ttm\color{codegray},
otherkeywords={self},             % Add keywords here
keywordstyle=\ttb\color{deepblue},
emph={MyClass,__init__},          % Custom highlighting
emphstyle=\ttb\color{deepred},    % Custom highlighting style
stringstyle=\color{deepgreen},
frame=none,                         % Any extra options here
showstringspaces=false,            % 
backgroundcolor=\transparent{.5}\color{backcodepy},
breaklines=true,
postbreak=\raisebox{0ex}[0ex][0ex]{\ensuremath{\color{red}\hookrightarrow\space}}
}}

\lstnewenvironment{python}[1][]
{
\pythonstyle
\lstset{#1}
}
{}

% Python for external files
\newcommand\pythonexternal[2][]{{
\pythonstyle
\lstinputlisting[#1]{#2}}}

% Python for inline
\newcommand\pythoninline[1]{{\pythonstyle\lstinline!#1!}} 
%%%%%%%%%%%%%%%%%%%%%%%%%%%%%%%%%%%%%%%%%%%%%%%%%%%%%%%%%%%%%%%%%%%%%%%%%%%%%%%%%%%%%%%%%%%%%%%%%%%%

%%%%%%%%%%%%%%%%%%%%%%%%%%%%%%%%%%%%%%%%%%%%%%%%%%%%%%%%%%%%%%%%%%%%%%%%%%%%%%%%%%%%%%%%%%%%%%%%%%%%
%% SQL enviorment
%%%%%%%%%%%%%%%%%%%%%%%%%%%%%%%%%%%%%%%%%%%%%%%%%%%%%%%%%%%%%%%%%%%%%%%%%%%%%%%%%%%%%%%%%%%%%%%%%%%%

\definecolor{forestgreen}{RGB}{34,139,34}
\definecolor{orangered}{RGB}{239,134,64}
\definecolor{darkblue}{rgb}{0.0,0.0,0.6}
\definecolor{gray}{rgb}{0.4,0.4,0.4}
\definecolor{light-gray}{gray}{0.95}
\lstdefinestyle{mySQL} {
    backgroundcolor=\color{backcolour},   
    language=SQL,
    extendedchars=true,
    breakatwhitespace=false,
    emph={},
    emphstyle=\color{red},
    basicstyle=\ttfamily\small,
    frame=tb,
    columns=fullflexible,
    commentstyle=\color{gray}\upshape,
    morestring=[b]",
    morecomment=[s]{<?}{?>},
    morecomment=[s][\color{forestgreen}]{<!--}{-->},
    keywordstyle=\color{blue}\ttfamily,
    stringstyle=\ttfamily\color{black}\normalfont,
    %tagstyle=\color{darkblue}\bf,
    otherkeywords={attribute=, xmlns=},
    breaklines=true,                 
    numbers=none, %other option is left
    escapechar=\%,
    rulecolor=\color{backframecolour},
    postbreak=\raisebox{0ex}[0ex][0ex]{\ensuremath{\color{red}\hookrightarrow\space}}
}
%%%%%%%%%%%%%%%%%%%%%%%%%%%%%%%%%%%%%%%%%%%%%%%%%%%%%%%%%%%%%%%%%%%%%%%%%%%%%%%%%%%%%%%%%%%%%%%%%%%%

%%%%%%%%%%%%%%%%%%%%%%%%%%%%%%%%%%%%%%%%%%%%%%%%%%%%%%%%%%%%%%%%%%%%%%%%%%%%%%%%%%%%%%%%%%%%%%%%%%%%
%% Verbatim enviorment
%%%%%%%%%%%%%%%%%%%%%%%%%%%%%%%%%%%%%%%%%%%%%%%%%%%%%%%%%%%%%%%%%%%%%%%%%%%%%%%%%%%%%%%%%%%%%%%%%%%%
\lstdefinestyle{mystyle}{
    backgroundcolor=\color{backcolour},   
    commentstyle=\color{codegreen},
    keywordstyle=\color{magenta},
    numberstyle=\tiny\color{codegray},
           stringstyle=\color{codepurple},
               basicstyle=\ttfamily\color{black}\normalfont,  
stringstyle=\ttfamily\color{black}\normalfont,  
    breakatwhitespace=false,         
    captionpos=b,                    
    keepspaces=true,                 
    numbers=left,                    
    numbersep=5pt,                  
    showspaces=false,        
    linewidth=6.00in,
    showstringspaces=false,
    showtabs=false,                  
    tabsize=2,
    breaklines=true,                 
    postbreak=\raisebox{0ex}[0ex][0ex]{\ensuremath{\color{red}\hookrightarrow\space}}
}
%%%%%%%%%%%%%%%%%%%%%%%%%%%%%%%%%%%%%%%%%%%%%%%%%%%%%%%%%%%%%%%%%%%%%%%%%%%%%%%%%%%%%%%%%%%%%%%%%%%%




%%%%%%%%%%%%%%%%%%%%%%%%%%%%%%%%%%%%%%%%%%%%%%%%%%%%%%%%%%%%%%%%%%%%%%%%%%%%%%%%%%%%%%%%%%%%%%%%%%%%
%% XML enviorment
%%%%%%%%%%%%%%%%%%%%%%%%%%%%%%%%%%%%%%%%%%%%%%%%%%%%%%%%%%%%%%%%%%%%%%%%%%%%%%%%%%%%%%%%%%%%%%%%%%%%
\definecolor{forestgreen}{RGB}{34,139,34}
\definecolor{orangered}{RGB}{239,134,64}
\definecolor{darkblue}{rgb}{0.0,0.0,0.6}
\definecolor{gray}{rgb}{0.4,0.4,0.4}
\lstdefinestyle{XML} {
    language=XML,
    extendedchars=true,
    breakatwhitespace=false,
    emph={},
    emphstyle=\color{red},
    basicstyle=\ttfamily,
    columns=fullflexible,
    commentstyle=\color{gray}\upshape,
    morestring=[b]",
    morecomment=[s]{<?}{?>},
    morecomment=[s][\color{forestgreen}]{<!--}{-->},
    keywordstyle=\color{orangered},
    stringstyle=\ttfamily\color{black}\normalfont,
    tagstyle=\color{darkblue}\bfseries,
    morekeywords={attribute,xmlns,version,type,release},
    otherkeywords={attribute=, xmlns=},
    breaklines=true,                 
    postbreak=\raisebox{0ex}[0ex][0ex]{\ensuremath{\color{red}\hookrightarrow\space}}
}


%%%%%%%%%%%%%%%%%%%%%%%%%%%%%%%%%%%%%%%%%%%%%%%%%%%%%%%%%%%%%%%%%%%%%%%%%%%%%%%%%%%%%%%%%%%%%%%%%%%%




%%%%%%%%%%%%%%%%%%%%%%%%%%%%%%%%%%%%%%%%%%%%%%%%%%%%%%%%%%%%%%%%%%%%%%%%%%%%%%%%%%%%%%%%%%%%%%%%%%%%
%%IQL enviorment
%%%%%%%%%%%%%%%%%%%%%%%%%%%%%%%%%%%%%%%%%%%%%%%%%%%%%%%%%%%%%%%%%%%%%%%%%%%%%%%%%%%%%%%%%%%%%%%%%%%%
\lstdefinestyle{IQL} {
    language=SQL,
    extendedchars=true,
    breakatwhitespace=false,
    emph={},
    emphstyle=\color{red},
    basicstyle=\ttfamily,
    columns=fullflexible,
    commentstyle=\color{gray}\upshape,
    morecomment=[s]{<?}{?>},
    morecomment=[s][\color{forestgreen}]{<!--}{-->},
    keywordstyle=\color{blue}\ttfamily,
    stringstyle=\ttfamily\color{black}\normalfont,
    tagstyle=color{darkblue}\bf,
    morekeywords={echo, simulation_duration, manage,table, insert},
    otherkeywords={attribute=, xmlns=},
    breaklines=true,                 
    numbers=none,
    escapechar=\%,
    postbreak=\raisebox{0ex}[0ex][0ex]{\ensuremath{\color{red}\hookrightarrow\space}}
}
%%%%%%%%%%%%%%%%%%%%%%%%%%%%%%%%%%%%%%%%%%%%%%%%%%%%%%%%%%%%%%%%%%%%%%%%%%%%%%%%%%%%%%%%%%%%%%%%%%%%

%%%%%%%%%%%%%%%%%%%%%%%%%%%%%%%%%%%%%%%%%%%%%%%%%%%%%%%%%%%%%%%%%%%%%%%%%%%%%%%%%%%%%%%%%%%%%%%%%%%%
%%Shell enviorment
%%%%%%%%%%%%%%%%%%%%%%%%%%%%%%%%%%%%%%%%%%%%%%%%%%%%%%%%%%%%%%%%%%%%%%%%%%%%%%%%%%%%%%%%%%%%%%%%%%%%
\lstdefinestyle{myshell} {
    language=SQL,
    extendedchars=true,
    breakatwhitespace=false,
    emph={},
    emphstyle=\color{red},
    basicstyle=\ttfamily,
    columns=fullflexible,
    commentstyle=\color{gray}\upshape,
    morecomment=[s]{<?}{?>},
    morecomment=[s][\color{forestgreen}]{<!--}{-->},
    keywordstyle=\color{blue}\ttfamily,
    stringstyle=\ttfamily\color{black}\normalfont,
    tagstyle=\color{darkblue}\bf,
    breaklines=true,                 
    numbers=none,
    escapechar=\%,
    postbreak=\raisebox{0ex}[0ex][0ex]{\ensuremath{\color{red}\hookrightarrow\space}}
}



\lstloadlanguages{sh,python}
\def\cmdinline{\lstinline[basicstyle=\ttfamily,language=sh,frame=single]}
%%
\lstnewenvironment{codepy}
{\lstset{language=python}}
{}


\newcommand{\question}[1]{\textcolor{blue}{#1}}

%%%%%%%%%%%%%%%%%%%%%%%%%%%%%%%%%%%%%%%%%%%%%%%%%%%%%%%%%%%%%%%%%%%%%%%%%%%%%%%%%%%%%%%%%%%%%%%%%%%%
%% caution/warning box
%%%%%%%%%%%%%%%%%%%%%%%%%%%%%%%%%%%%%%%%%%%%%%%%%%%%%%%%%%%%%%%%%%%%%%%%%%%%%%%%%%%%%%%%%%%%%%%%%%%%

\renewenvironment{leftbar}[1][\hsize]%
{%
        \def\FrameCommand%
        {%
             \includegraphics[width=1cm]{latex_utils/handarrow.png}%           
             \fboxsep=\FrameSep\colorbox{cyan!5}%
        }%
        \MakeFramed{\hsize#1\advance\hsize-\width\FrameRestore}%
}%
{\endMakeFramed}
%%%%%%%%%%%%%%%%%%%%%%%%%%%%%%%%%%%%%%%%%%%%%%%%%%%%%%%%%%%%%%%%%%%%%%%%%%%%%%%%%%%%%%%%%%%%%%%%%%%%


%%%%%%%%%%%%%%%%%%%%%%%%%%%%%%%%%%%%%%%%%%%%%%%%%%%%%%%%%%%%%%%%%%%%%%%%%%%%%%%%%%%%%%%%%%%%%%%%%%%%
%% format for tables
%%%%%%%%%%%%%%%%%%%%%%%%%%%%%%%%%%%%%%%%%%%%%%%%%%%%%%%%%%%%%%%%%%%%%%%%%%%%%%%%%%%%%%%%%%%%%%%%%%%%
%\rowcolors{2}{gray!25}{pink}




\usepackage{xcolor}
\usepackage{url}
\usepackage{verbatim}
\usepackage{fixme}
\usepackage{listings}
\usepackage{graphicx}
\usepackage{subcaption}

\usepackage{listings,color}
\usepackage{soul}
\definecolor{verbgray}{gray}{0.9}
\definecolor{bgcol}{rgb}{.1, 1.0, 1.0}

\lstnewenvironment{code}{%
  \lstset{backgroundcolor=\color{verbgray},
  frame=single,
  framerule=0pt,
  basicstyle=\ttfamily,
  columns=fullflexible}}{}

\definecolor{shadecolor}{rgb}{.5, .5, .5}
\usepackage[margin=0.5in]{geometry}



%Experiment design can be arbitary choice of parameters
\title{Dicex-Epistudy: Design, Run, and, Analysze an Epidemiological Study}
%% \author{Gupta, Sandeep\\
%%     \texttt{sandeep@vbi.vt.edu}
%% \and
%%   Cedeño, Vanessa\\
%%   \texttt{vanessaicm@gmail.com}
%% }

\begin{document}
\maketitle
\tableofcontents

\chapter{Running a study}
\begin{enumerate}
\item Download (or update) epistudy\_configs from git repository (https://ndsslgit.vbi.vt.edu/ndssl-software/dicex\_epistudy) 
\begin{code}
you@sfxlogin1: cd <git_repo_dir>
you@sfxlogin1: git clone git@ndsslgit.vbi.vt.edu:ndssl-software/dicex_epistudy.git
\end{code}
\item Turn on the dicex enviorment
\begin{code}
you@sfxlogin1:. /home/pgxxc/public/dicex/dicex.sh
\end{code}
\emph {The dot in the last command is necessary}
\item Create an experiment directory from where you will run a study
\begin{code}
you@sfxlogin1: mkdir <exp_work_dir>
you@sfxlogin1: cd <exp_work_dir>
\end{code}
\item Choose a study config file from epistudy\_configs, lets say block study. The config file for this study is in epistudy\_configs/block/epistudy\_cfg.xml
\item Copy the file to the experiment work directory
\begin{code}
you@sfxlogin1: cd <exp_work_dir>
you@sfxlogin1: cp <git_repo_dir>/epistudy_configs/block/epistudy_cfg.xml .
\end{code}

\item Open the file and modify the paramters. The parameters mentioned at the top (shown below) of the file is mandatory 
\begin{code}
<?xml version="1.0" encoding="UTF-8"?>
<!DOCTYPE epistudy [
<!ENTITY var_region "Miami">
<!ENTITY var_cfg_dir "/home/sandeep/git_repos/epistudy_configs/">
<!ENTITY var_run_dir "/home/sandeep/experiments/dicex_epistudy/">
<!ENTITY var_intv_type "block">
<!ENTITY var_dicex_base_dir "/home/pgxxc/public/">
...
...
]>
\end{code}
Change the var\_cfg\_dir, var\_run\_dir to point to appropriate directory. var\_cfg\_dir is 
the path for epistudy\_configs directory. var\_run\_dir is the path for directory where the
run output and config files will be stored. 
\item Now call epistudy\_workflow with the corresponding epistudy\_config file. For block study the command
would look like
\begin{code}
you@sfxlogin1: epistudy_workflow.sh epistudy_cfg.xml
\end{code}
This would launch a database, and run all the  cells, and the replicates. 
\end{enumerate}

\paragraph{Some additional fine points and dicex intracacies/weirdness/gotchas}

\begin{itemize}
\item {\bf The database runs on compute node on an arbitary chosen port}. To access the database, first find the hostname and port number from the  dbsession\_$<$dbsession-tag$>$.py file.  This file would look as follows:
\begin{code}
server_ip="sfx052"
server_port=5853
work_dir="/home/sandeep/experiments/dicex_epistudy/tmpGn417y"
jobid="1966115.master.cm.cluster"
pgpid="4219"
\end{code}
Now ssh into the compute node as 
\begin{code}
you@sfxlogin1: ssh <server_ip>   #or in this example, ssh sfx052
\end{code}

Access the database as follows:
\begin{code}
you@<some-compute-node>: psql -p <server_port> postgres  #or in this example,  psql -p 5853 postgres
\end{code}
Alternatively, the tables can be accessed from the shadowfax login node itself:
\begin{code}
you@sfxlogin1: dpsql epistudy_cfg.xml
\end{code}
This will open the postgres client program and provide a prompt. You would need to know basic 
sql (\url{http://www.itl.nist.gov/div897/ctg/dm/sql_examples.htm}).

\item {\hl {\emph{\bf Please remove the job after it is finished}}}\\
This is necessary because we create files in the temporary disk folder and if left unremoved it would cause troubles. lease follow these steps. 
\begin{code}
you@sfxlogin1: epistudy_cleanup.sh epistudy_cfg.xml
\end{code}

\end{itemize}

\chapter{Simulations, Interventions, and Case Studies Using Indemics}
\begin{itemize}
\item cells
\item replicates
\item study
\item iql script
\item base tables
\item dynamic tables
\item efficacies
\item compliance
\item delay
\end{itemize}


\chapter{Epistudy Architecture}

\chapter{Designing a case study}
A case study consists of contagion simulation, a set of interventions and  a factorial experiment design. Each intervention
also has an associated trigger that dictates when the intervention will be applied.
Dicex\_epistudy requires three inputs to describe an study, namely, a base data config file, an iql template file,  and a master
design file.
Base data config file describes the base datasets required to run an intervention. The iql template file describes the triggers
and the intervention as series of SQL statements to the indemics client.
The mater design file is a config file that describes the factorial design and the  rest of the system level parameters for the case study.

\subsection{The base data config file}
The base data config file is an dicex external data input configuration file. The example below
shows the config file for base datasets for block intervention which consists of block\_info and
block\_intervened datasets.
\begin{lstlisting}[style=XML]
  <edcfg>
  <files>
    <file>
      <location>/groups/NDSSL/indemics_using_pgxc/&var_region;/block_info.txt</location>
      <metadata>&var_metadata_dir;/block_info.md</metadata>
    </file>
    <file>
      <location>/groups/NDSSL/indemics_using_pgxc/&var_region;/block_intervened.txt</location>
      <metadata>&var_metadata_dir;/block_intervened.md</metadata>
    </file>
  </files>
</edcfg>
\end{lstlisting}
This configuration file lists the location and the metadata for the two base datasets  for the block
intervention.
The variables \&var\_region and \&var\_metadata\_dir are defined in the master config.

\subsection{The template iql file}
The template iql file is the indemics script file but using variables for replicate identifiers
and intervention parameters instead of values. The dicex-epistudy tool uses the template
file to create iql file for each cell and replicates. Below shows the iql template file
for block intervention:
\begin{lstlisting}[style=IQL]
echo "set player id: ndssl,ndssl"

echo %"%select session: session = ${replicate_desc}, super player%"%
echo %"%manage table : create index ${replicate_desc}_diag_pid_idx on &{diagnosed table} (pid)%"%

echo %"%manage table : insert into ${replicate_desc}_block_intervened select bid, persons, -1 from block_intervened%"%

for(( i=0; i < ${simulation_duration} ; i++ ))
do

	echo %"%manage table: insert into ${replicate_desc}_block_daily_diag_count select b.block, count(b.pid), $i from block_info b, &{diagnosed table} d where time = $i and  b.pid = d.pid group by block%"%


        echo %"%manage table: insert into ${replicate_desc}_block_win_diag_count select bid, sum(infections), $i from ${replicate_desc}_block_daily_diag_count  where day between `expr $i - ${i0_daysill}` and $i group by bid%"%

	 echo %"%manage table: update ${replicate_desc}_block_intervened set intervened = $i from (select s.bid from ${replicate_desc}_block_win_diag_count s, ${replicate_desc}_block_intervened i where day = $i and (s.infections::float/i.persons > ${i0_threshold}) and s.bid=i.bid) AS O where intervened = -1 and ${replicate_desc}_block_intervened.bid=O.bid%"%

	echo %"%set interventions:  0,${i0_type},${i0_compliance},${i0_duration},${i0_delay},${i0_efficacy_in},${i0_efficacy_out}, select pid from block_info b, ${replicate_desc}_block_intervened i where b.block=i.bid and i.intervened = $i%"%

done

echo "stop session:"
echo "stop client:"

\end{lstlisting}
The iql file is a series of statements ending with a ''set intervention'' statement. 
The terms such as \$\{replicate\_desc\}\$, \$\{i0\_daysill\}\$ are templated variables
that are filled from the master config file.

\subsection{The master config file}
The master config file describes system parameters and experiment design of the study.
Following is an sample example.
\begin{lstlisting}[style=XML]
  <?xml version="1.0" encoding="UTF-8"?>
<!DOCTYPE epistudy [
<!ENTITY var_region "Miami">
<!ENTITY var_cfg_dir "/home/pgxxc/public/epistudy_configs/">
<!ENTITY var_run_dir "~/experiments/dicex_epistudy/">
<!ENTITY var_intv_type "block">
<!ENTITY var_dicex_base_dir "/home/pgxxc/public/">
<!ENTITY var_metadata_dir "/groups/NDSSL/dicex_metadata_repo/">
<!ENTITY var_epifast_socnet_path_template "/groups/ndsslpub/isis_areas/${study_region}/socialnet/EFIG6Bb">
]>

<epistudy>
  <title>&var_region;_&var_intv_type;</title>
  <region>&var_region;</region>
  <study_base_dir>&var_run_dir;/&var_region;_&var_intv_type;</study_base_dir>
  <epistudy_configs_dir>&var_cfg_dir;</epistudy_configs_dir>
  <iql_template>&var_cfg_dir;/&var_intv_type;/iql_&var_intv_type;.template</iql_template>

  <model_data>
    <edcfg_file>&var_cfg_dir;/&var_intv_type;/intv_base_dataset_edcfg.xml</edcfg_file>
  </model_data>

  <interventions>
    <intervention0>
      <iql_key>i0</iql_key>
      <type>0</type>
      <compliance>1.0</compliance>
      <duration>300</duration>
      <delay>20</delay>
      <efficacy_in>0.1</efficacy_in>
      <efficacy_out>0.1</efficacy_out>
      <daysill>1</daysill>
      <threshold>0.06</threshold>
    </intervention0>
  </interventions>

  <sweep> 
    <parameter>intervention0/compliance</parameter>
    <abbrv>c</abbrv>
    <type>strided</type>
    <start_val>.30</start_val>
    <inc>.10</inc>
    <end_val>.40</end_val>
    <sweep>
      <parameter>replicate</parameter>
      <abbrv>r</abbrv>
      <type>strided</type>
	<start_val>0</start_val>
	<inc>1</inc>
	<end_val>1</end_val>
	<pre_intv_begin>
          <create_table>
            <metadata>&var_metadata_dir;/infections.md</metadata>
            <index_column>day</index_column>
          </create_table>
	  <create_table>
	    <metadata>&var_metadata_dir;/block_daily_diag_count.md</metadata>
	    <index_column>day</index_column>
	  </create_table>
	  <create_table>
	    <metadata>&var_metadata_dir;/block_win_diag_count.md</metadata>
	    <index_column>day</index_column>
	  </create_table>
	  <create_table>
	    <metadata>&var_metadata_dir;/block_intervened.md</metadata>
	    <index_column>bid</index_column>
	  </create_table>
	</pre_intv_begin>

    </sweep>
  </sweep>
  
  <post_study_end>
    <analysis>
      
    </analysis>
  </post_study_end>


  <epifast>
    <socnet_path_template>&var_epifast_socnet_path_template;</socnet_path_template>
    <num_nodes>8</num_nodes>
    <walltime>1</walltime>
    <Diagnosis> 
      <ModelVersion>2009</ModelVersion>
      <ProbSymptomaticToHospital>1.0</ProbSymptomaticToHospital>
      <ProbDiagnoseSymptomatic>0.667</ProbDiagnoseSymptomatic>
      <DiagnosedDuration>6</DiagnosedDuration>
      </Diagnosis>
    <transmissibility>0.0000535</transmissibility>
    <asymptomatic_probability>0.3333</asymptomatic_probability>
    <asymptomatic_factor>0.5</asymptomatic_factor>
    <simulation_duration>300</simulation_duration>
    <seed>12345</seed>
  </epifast>

  <system>
    <template_dir>templates</template_dir>  
    <indemics_dir>&var_dicex_base_dir;/indemics_server</indemics_dir>
    <dicex_base_dir>&var_dicex_base_dir;</dicex_base_dir>
  </system>


  <data_engine>
    <engine_type>pgsa</engine_type>
    <dbsession>&var_region;_&var_intv_type;</dbsession>
    <walltime>1</walltime>
  </data_engine>

  <indemics_server>
    <max_threads>8</max_threads>
    <walltime>1</walltime>
  </indemics_server>
</epistudy>
\end{lstlisting}
%TODO describes various parts of the file


\subsection{The factorial design}
Dicex\_epistudy supports constructs to design a  study, i.e., define the parameters for  cells (and the number of replicates). It consists of two portions: the default parameter values and the (nested) parameter  set of values.
For example, the following XML snippets describes the settings for default values of the parameters:
\begin{lstlisting}[style=XML]
    <interventions>
    <intervention0>
      <iql_key>i0</iql_key>
      <type>0</type>
      <compliance>1.0</compliance>
      <duration>300</duration>
      <delay>20</delay>
      <efficacy_in>0.1</efficacy_in>
      <efficacy_out>0.1</efficacy_out>
      <daysill>1</daysill>
      <threshold>0.06</threshold>
    </intervention0>
  </interventions>
  \end{lstlisting}


For example, the following XML snippets encode a design over block intervention.
\begin{lstlisting}[style=XML]
  <sweep> 
    <parameter>intervention0/compliance</parameter>
    <abbrv>c0</abbrv>
    <type>strided</type>
    <start_val>.30</start_val>
    <inc>.10</inc>
    <end_val>.40</end_val>
    <sweep>
      <parameter>replicate</parameter>
      <abbrv>r</abbrv>
      <type>strided</type>
	<start_val>0</start_val>
	<inc>1</inc>
	<end_val>1</end_val>
	<pre_intv_begin>
          <create_table>
            <metadata>&var_metadata_dir;/infections.md</metadata>
            <index_column>day</index_column>
          </create_table>
	  <create_table>
	    <metadata>&var_metadata_dir;/block_daily_diag_count.md</metadata>
	    <index_column>day</index_column>
	  </create_table>
	  <create_table>
	    <metadata>&var_metadata_dir;/block_win_diag_count.md</metadata>
	    <index_column>day</index_column>
	  </create_table>
	  <create_table>
	    <metadata>&var_metadata_dir;/block_intervened.md</metadata>
	    <index_column>bid</index_column>
	  </create_table>
	</pre_intv_begin>

    </sweep>
  </sweep>
\end{lstlisting}
The primary keyword to define a  factorial design is ''sweep''. Each sweep element defines
span of the values for parameter with which to run the simulation.
The sweep element can be nested, which defines the factorial design.
In the example, the top most parameter to sweep is ''intervention0/compliance'',
starting with $.30$ and ending value of $.40$ with increment of $.1$.
The element abbrv is an abbriviation used as descriptor for cells and replicate.

The next level nested element defines sweep on replicate parameter which is a special
parameters that defines the number replicate per cell.
The above design creates two cells, one for each value of compliance parameter, namely $0.3$ and
$0.$. For each of the cells, there are two replicates.
The descriptor for replicates are respectively:
$c0\_0.3\_r0,c0\_0.3\_r1,c0\_0.4\_r0,c0\_0.4\_r1$.





  
\chapter{Analytics and Visualization on replicate simulation outputs}
In addition to defining an experiment design and running it using Shadowfax, Dicex\_EpiStudy allows for analysis and visualization of output data across the replicates of the experiment design. 
The main functionality here is that the  configuration file used to define the experiment
design is also used for driving the visualization and analysis. 

Provided with this git repository is a set of analysis scripts for block intervention study. These are
\begin{itemize}
\item replicate\_analysis\_block\_intv.py
\item replicate\_analysis\_block\_intv\_impl.py
\end{itemize}

Browse through these files to get a feel how to use the analysis routines. 


\subsection{Running an analysis for each replicate}
Running a single type of analysis for each of the replicate is done through a mechanism that is often described as visitor pattern. 
The analysis is described as function that has the signature described below. 
Next, this function is handed over to another function called visit\_replicates.
visit\_replicates calls the analysis function for each occurance of the replicate descriptor. 
The signature of the analysis function
\begin{code}
def my_analysis_func(cfg_root=None, param_root=None, replicate_desc=None, rep_dir=None, args=None):
\end{code}
Here the main arguments are replicate\_desc which is the descriptor of the replicate and
the args. 

Now, to run the analysis for the each of the generated replicate call the visitor with your analysis function as argument as follows:
\begin{verbatim}
import replicate_visitor as rv
rv.visit_replicates(cfg_root=cfg_root, method=my_analysis_func, args=[session])
\end{verbatim}
The last ``[session]'' argument is necessary. If my\_analysis\_func needs additional arguments, then they should be added to the args list. For a complete working example see the accompanying example files mentioned above. 

\paragraph{\question{Why use visitor pattern and not simply a list of replicate descriptors?}}\hfill
The descriptor names are ``opaque'' and do not tell what the configurations of the replicates where. The cfg\_root is a handle to the XML file the defines the replicate configuration. It can be used, for example, to normalize the output value with respect to input parameter value.

\subsection{What kind of analysis/plots are possible using the tool}
The Dicex\_Epistudy uses VERSA data engine for analysis and plotting. The possibilites are in the relam of data joining, filtering, and aggregations (count, sum, mean, max, avg.). Currently, line and bar graphs are supported (a cool Dynamic-Circos feature should be added soon).

However, lets go over the simplest analysis where we just print the output on to the console, sort of the ``Hello World'' equivalent. 

\begin{verbatim}
def hello_world(cfg_root=None, param_root=None, replicate_desc=None, rep_dir=None, args=None):
    import versa_api as vapi
    session = args[0]
    model_obj = rau.get_replicate_model_obj(replicate_desc, '_block_daily_diag_count') 
    print vapi.scan(session, model_obj)
\end{verbatim}

The above will print the records for ''block\_daily\_diag\_count'' table for all the replicates. 

\subsection{Drawing epicurves for each replicates}
The table ''block\_daily\_diag\_count'' maintains number of infections per day per block. 
In the next example, we will aggregate over day to compute number of infections per day and plot this generated epicurve. 

\begin{verbatim}
import versa_api as vapi
import versa_api_analysis as vapia
def draw_epicurve(cfg_root=None, param_root=None, replicate_desc=None, rep_dir=None, args=None):

    session = args[0]
    model_obj = rau.get_replicate_model_obj(replicate_desc, '_block_daily_diag_count') 
    res = vapia.build_distribution_X_vs_agg_Y(session = session, model =model_obj,  agg_by= 'day', agg_on= 'infections')
    vapip.plot_X_vs_Y(res, replicate_desc+'infections_per_day', 'day', '#infections')
    print vapi.scan(session, model_obj)
\end{verbatim}
The function build\_distribution\_X\_vs\_agg\_Y computes the aggregate on 'day' by summing across all the block infections. It returns the result dictionary with two keys: xValues and yValues. 
The function plot\_X\_vs\_Y plots this epicurve. 

\begin{figure}
\begin{subfigure}{.5\textwidth}
  \centering
  \includegraphics[width=.8\linewidth]{c_08_r_0infections_per_day.pdf}
  \caption{c\_08r\_0}
  \label{fig:sfig1}
\end{subfigure}%
\begin{subfigure}{.5\textwidth}
  \centering
  \includegraphics[width=.8\linewidth]{c_08_r_1infections_per_day.pdf}
  \caption{c\_08r\_1}
  \label{fig:sfig2}
\end{subfigure}
\begin{subfigure}{.5\textwidth}
  \centering
  \includegraphics[width=.8\linewidth]{c_09_r_0infections_per_day.pdf}
  \caption{c\_09r\_0}
  \label{fig:sfig2}
\end{subfigure}
\begin{subfigure}{.5\textwidth}
  \centering
  \includegraphics[width=.8\linewidth]{c_09_r_0infections_per_day.pdf}
  \caption{c\_09r\_1}
  \label{fig:sfig2}
\end{subfigure}
\caption{Performing analysis and  plots for each replicate}
\label{fig:fig}
\end{figure}


To reiterate, if we pass the function as argument to  visit\_replicate as follows:
\begin{verbatim}
rv.visit_replicates(cfg_root=cfg_root, method=my_analysis_func, args=[session])
\end{verbatim}
it would compute and plot epicurve for each of the replicates. 

\subsection{Analysis using data from all replicates}


\end{document}
